\documentclass[12pt]{article}
\begin{document}

\begin{titlepage}\begin{center}\Large MIPT ICT, 2020\end{center}\begin{center}\Large\textit{Belyaev Andrey Alexeevich}\end{center}\begin{center}\Huge\textbf{\underline{D I F F E R E N T I A T I O N}}\end{center}\begin{center}\large\textit{Patent pending}\end{center}\section{\Large{Abstract}}So, scientific works get payed for, right? I've got a master plan...\end{titlepage}

Thankfully, the derivative is additive.

$$ \frac{d}{dx} \left( x  +  7  \cdot \cos  y ^{ x ^{ 2 }}\right) = \left( x \right)' + \left( 7  \cdot \cos  y ^{ x ^{ 2 }}\right)' $$

Multiplication is a bit difficult, but we can still manage it.

$$ \frac{d (f \cdot g)}{dx} = f' \cdot g + f \cdot g' $$
$$ f(x) =  7  $$
 $$ g(x) = \cos  y ^{ x ^{ 2 }} $$

Exponential differentiation is not as scary as it looks.

$$ \frac{d}{dx} a^f = \ln{a} \cdot a^f \cdot f' $$
$$ f(x) = \cos  y  $$
 $$ a =  x ^{ 2 } $$

Polynomial differentiation is a piece of cake.

$$ \frac{d}{dx} f^{a} = a \cdot f^{a - 1} \cdot f' $$
$$ f(x) =  x  $$
 $$ a =  2  $$

The derivative of the target variable is obvious.

$$ \frac{dx}{dx} = 1 $$

So, this subexpression results in:

$$  2  \cdot  x ^{ 2  -  1 } \cdot  1  $$

If we simplify it, we get:

$$  2  \cdot  x  $$

So, this subexpression results in:

$$ \ln \cos  y  \cdot \cos  y ^{ x ^{ 2 }} \cdot  2  \cdot  x  $$

Differentiation of a constant is trivial.

$$ \frac{d}{dx} 7 = 0 $$

So, this subexpression results in:

$$  0  \cdot \cos  y ^{ x ^{ 2 }} +  7  \cdot \ln \cos  y  \cdot \cos  y ^{ x ^{ 2 }} \cdot  2  \cdot  x  $$

If we simplify it, we get:

$$  7  \cdot \ln \cos  y  \cdot \cos  y ^{ x ^{ 2 }} \cdot  2  \cdot  x  $$

The derivative of the target variable is obvious.

$$ \frac{dx}{dx} = 1 $$

So, this subexpression results in:

$$  1  +  7  \cdot \ln \cos  y  \cdot \cos  y ^{ x ^{ 2 }} \cdot  2  \cdot  x  $$

\end{document}

